\documentclass[openany, oneside,]{book}
% pre\'ambulo
\textheight = 22.5cm
\textwidth = 13.5cm
\topmargin = -1.5cm
\oddsidemargin = 1.5cm

\newtheorem{teo}{Teorema}[chapter]
\newtheorem{defi}[teo]{Definici\'on}
\newtheorem{cor}[teo]{Corolario}
\newtheorem{pro}[teo]{Proposici\'on}
\newtheorem{lem}[teo]{Lema}
\newtheorem{dem}[teo]{Demostraci\'on}

\renewcommand{\contentsname}{Contenido}
\renewcommand{\partname}{Parte}
\renewcommand{\appendixname}{Apéndice}
\renewcommand{\figurename}{Figura}
\renewcommand{\tablename}{Tabla}
\renewcommand{\chaptername}{Capítulo} % para ’book’
\renewcommand{\bibname}{Bibliografía} % para ’book’

\usepackage{lmodern}
\usepackage[latin1]{inputenc}
\usepackage[spanish,activeacute]{babel}
\usepackage{mathtools,amssymb,latexsym,amsmath,amsfonts}
\usepackage{makeidx}
\usepackage{fancyhdr}
\usepackage{graphicx}

% encabezados
\lhead[\rightmark]{\small\leftmark}
\chead[]{}
\rhead[]{\thepage}
\renewcommand{\headrulewidth}{0.5pt}

% pie de pagina
\lfoot[]{\rm Par Biomagn\'etico (Austin, TX) \\  Derechos Reservados {\copyright} 2014 Biomagnetismo - Goiz (http://www.biomagnetismo.biz)}
\cfoot[]{}
\rfoot[]{David Goiz Mart\'inez BMB01EU001}
\renewcommand{\footrulewidth}{0.5pt}

% primera pagina de un capitulo
\fancypagestyle{plain}{
\fancyhead[L]{}
\fancyhead[C]{}
\fancyhead[R]{\thepage}
\fancyfoot[L]{}
\fancyfoot[C]{}
\fancyfoot[R]{}
\renewcommand{\headrulewidth}{0pt}
\renewcommand{\footrulewidth}{0pt}}


\pagestyle{fancy}

\begin{document}
\title {\Huge Par Biomagn\'etico \\ (Austin,TX)}
\author{\Large Dr. David Goiz Mart\'inez\\}

\date{\Large 29 Julio - 03 Agosto 2014}

% cuerpo del documento

\maketitle

\begin{center}
\subsection*{Lista de Pares Biomagn'eticos}

\begin{tabular}{|p{5cm}|p{3.5cm}|p{3.5cm}|}\hline

\bf \begin{center}
Patolog'ia
\end{center} & \bf\begin{center}
 Punto de Rastreo
\end{center} & \bf \begin{center}
Punto de Impacto
\end{center} \\\hline

\begin{center}
1.- Infecci'on Viral
 \par (VIH)\end{center} & \centering \begin{center}
 Timo
 \end{center} & {\centering \begin{center}
 Recto
 \end{center}} \\\hline

\end{tabular}
\end{center}

\end{document}