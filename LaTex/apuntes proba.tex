\documentclass[]{article}
\usepackage[spanish]{babel}
    \usepackage{tikz}



%opening
\title{Apuntes Probabilidad}
\author{Adr\'ian Mendoza}

\begin{document}
\maketitle

La probabilidad de un suceso es un n\'umero, comprendido entre 0 y 1, que indica las posibilidades que tiene de verificarse cuando se realiza un experimento aleatorio.\\\\
Teor\'ia de Probabilidades\\\\
La teor\'ia de probabilidades se ocupa de asignar un cierto n\'umero a cada posible resultado que pueda ocurrir en un experimento aleatorio, con el fin de cuantificar dichos resultados y saber si un suceso es m\'as probable que otro.Con este fin, introduciremos las siguientes definiciones:\\\\
Suceso: es cada uno de los resultados posibles de una experiencia aleatoria.\\\\
Espacio Muestral: es el conjunto de todos los posibles resultados de una experiancia aleatoria. Se denota como "$\Omega$".\\\\
Suceso Aleatorio: es cualquier subconjunto del espacio muestral.\\\\
El espacio muestral asociado con un experimento es el conjunto formado por todos los posibles puntos muestrales. Un espacio muestral se denota : S.\\\\
Un espacio muestral discreto es aquel que esta formado ya sea por un n\'umero finito o uno contable de puntos muestrales distintos.\\
Un evento en un espacio muestral discreto S es un conjunto de puntos muestrales , es decir, cualquier subconjunto de S.\\\\
\newpage
Tenemos que para la probabilidad de A ,P(A) se deben cumplir los siguientes axiomas:\\\\
Axioma 1.
\begin{center}
	 P(A) $\geq 0$ .
\end{center}
 Axioma 2.
\begin{center}
	 P(S) = 1.
\end{center}
 Axioma 3.
\begin{center}
	 Si $A_{1}$,$A_{2}$,$A_{3}$...  forman una secuencia de eventos por pares mutuamente excluyentes en S es decir , $A_{i}\cap A_{j} = 0$ si $i \neq j$ entonces:\\
 $P(A_{1}\cup A_{2}\cup A_{3}\cup...)$ = $\displaystyle
 \sum_{i=1}^{\infty}P(A_{i})$
\end{center}

Ejemplo:\\\\
Una bolsa contiene bolas blancas y negras. Se extraen sucesivamente tres bolas.\\
Calcular: a) espacio muestral, b) suceso A, c) suceso B, d) suceso C\\\\
a) el espacio muestral lo obtenemos con el conjunto potencia esto es P(A) o $2^{A}$, entonces siendo A=3 tenemos:\\\\
$2^{3}=8$\\
(b,b,b), (b,b,n), (b,n,n), (b,n,b), (n,b,b), (n,n,b), (n,b,n), (n,n,n)\\\\
b) extraer tres bolas del mismo color \\
(b,b,b), (n,n,n)\\\\
c) extraer al menos una bola blanca \\
(b,b,b), (b,b,n), (b,n,n), (b,n,b), (n,b,b), (n,n,b), (n,b,n)\\\\
d)extraer una sola bola negra\\
(b,n,b), (n,b,b), (b,b,n) .
 
 \newpage

Pasos para hallar la probabilidad de un evento:\\\\
1. Defina el experimento y determine con toda claridad c\'omo describir un evento simple.\\\\
2. Indique los eventos simples asociados con el experimento y prueb cada uno para asegurarse que no se pueden decomponer. Esto define el espacio muestral S.\\\\
3. Asigne probabilidades razonables a los puntos muestrales en S, asegur\'andose de que $P(E_{i}) \geq 0$ y $\sum P(E_{i}) = 1$ \\\\
4. Defina un evento de inter\'es , A, como un conjunto espec\'ifico de puntos muestrales . \\(Un punto muestral est\'a en A si A ocurre cuando se presenta el punto muestral. "Pruebe todos los puntos muestrales en S para indicar que est\'an en A").\\\\
5.Encuentre P(A) al sumar las posibilidades de los puntos muestrales en A. 




\end{document}
