\documentclass[12pt,a4paper,scrartcl]{article}

\usepackage{amsmath}
\usepackage[spanish]{babel}
\usepackage[left=2cm,right=2cm,top=2cm,bottom=2cm]{geometry}


\title{	
\normalfont \normalsize 
\textsc{Instituto Polit\'ecnico Nacional\\
Escuela Superior de F\'isica y Matem\'aticas} \\
}

\author{Adri\'an Mendoza Zamora\\
Juan Carlos Cervantes\\
Genaro C\'ordova \\
Equipo 12} 
\begin{document}

\maketitle

{\textbf{{Ejercicios 1er Parcial  An\'alisis Matem\'atico }}

\textbf{Secci\'on 1.1}\\

%seccion 1.1
1.Desarrollar $(a+b)^n$\\

con n=0\\
\begin{center}
$(a+b)^0=1$
\end{center}

con n=1\\
\begin{center}
$(a+b)^1=a+b$
\end{center}

con n=2\\
\begin{center}
$(a+b)^2=a^2+2ab+b^2$
\end{center}

con n=3\\
\begin{center}
$(a+b)^3=a^3+3a^2b+3ab^2+b^3$
\end{center}

\begin{center}
$\vdots$
\end{center}

\begin{center}
$(a+b)^n=\binom{n}{0}a^nb^0+\binom{n}{1}a^{n-1}b+\binom{n}{2}a^{n-2}b^2+\dots+\binom{n}{r}a^{n-r}b^r+\dots+\binom{n}{n-1}ab^{n-1}+\binom{n}{n}a^0b^n$
\end{center}





\newpage
2. Demostrar para $a,b$ n\'umeros reales y $n$ n\'umero natural que:\\

%inciso a
a) 
$a^n-b^n = (a-b)(a^{n-1}+a^{n-2}b+a^{n-3}b^2+ \hdots +a^{2}b^{n-3}+ab^{n-2}+b^{n-1}) $\linebreak
\begin{center}
$a^n-b^n=a^n-a^{n-1}b+a^{n-1}b-a^{n-2}b^2+a^{n-2}b^2-a^{n-3}b^3+\ldots+\linebreak+a^3b^{n-3}-a^{2}b^{n-1}+a^2b^{n-2}-ab^{n-1}+ab^{n-1}-b^n$
\end{center}
Por lo tanto \\
\begin{center}
$a^n-b^n=a^n-b^n$.
\end{center}
\vspace{5em}

%inciso b
b) $a-b = 
(a^{\frac{1}{n}}-b^{\frac{1}{n}})(a^{\frac{n-1}{n}}+a^{\frac{n-2}{n}}b^{\frac{1}{n}}+a^{\frac{n-3}{n}}b^{\frac{2}{n}}+ \hdots +a^{\frac{2}{n}}b^{\frac{n-3}{n}}+a^{\frac{1}{n}}b^{\frac{n-2}{n}}+b^{\frac{n-1}{n}})$\linebreak
\begin{center}
$a-b=
a^{\frac{1+n-1}{n}}+a^{\frac{1+n-2}{n}}b^{\frac{1}{n}}+a^{\frac{1+n-3}{n}}b^{\frac{2}{n}}+\ldots+a^{\frac{1+2}{n}}b^{\frac{n-3}{n}}+a^{\frac{1+1}{n}}b^{\frac{n-2}{n}}+a^{\frac{1}{n}}b^{\frac{n-1}{n}}-a^{\frac{n-1}{n}}b^{\frac{1}{n}}-a^{\frac{n-2}{n}}b^{\frac{1+1}{n}}-a^{\frac{n-3}{n}}b^{\frac{2+1}{n}}-\ldots-a^{\frac{2}{n}}b^{\frac{n-3+1}{n}}-a^{\frac{1}{n}}b^{\frac{n-2+1}{n}}-b^{\frac{n-1+1}{n}}$
\end{center}

\begin{center}
$a-b=a+a^{\frac{n-1}{n}}b^{\frac{1}{n}}+a^{\frac{n-2}{n}}b^{\frac{2}{n}}+\ldots+a^{\frac{3}{n}}b^{\frac{n-3}{n}}+a^{\frac{2}{n}}b^{\frac{n-2}{n}}+a^{\frac{1}{n}}b^{\frac{n-1}{n}}-a^{\frac{n-1}{n}}b^{\frac{1}{n}}-a^{\frac{n-2}{n}}b^{\frac{2}{n}}-a^{\frac{n-3}{n}}b^{\frac{3}{n}}-\ldots-a^{\frac{2}{n}}b^{\frac{n-2}{n}}-a^{\frac{1}{n}}b^{\frac{n-1}{n}}-b$\\
\end{center}
\begin{flushleft}
As\'i 
\end{flushleft}\
\begin{center}
$a-b=a-b$.
\end{center}





\newpage
3.Verificar las siguientes igualdades:\\
\\
$
\displaystyle
a)  \sum_{k=1}^{n}k = \dfrac{n(n+1)}{2} \\
$
\begin{center}
$S=1+2+3+\ldots+(n-1)+n$\\$\underline{S=n+(n+1)+\ldots+3+2+1}$\\$2S=(n+1)+(n+1)+\ldots+(n+1)$\\\vspace{1em}
$2S=n(n+1)$\\\vspace{1em}
\begin{center}
$S=\dfrac{n(n+1)}{2}\textbf{.}$
\end{center}
\end{center}Por inducci\'on:\\
\\I. La f\'ormula cumple para n=1, porque:\\
\begin{center}
$1=\dfrac{1(1+1)}{2}=1$
\end{center}
II. Hip\'otesis\\Entonces, si la f\'ormula cumple para n=k, tambien debe serlo para k+1 :
\begin{center}
$1+2+\ldots+k=\dfrac{k(k+1)}{2}$\\
\end{center}
III. Tesis\\
\begin{center}
$1+2+\ldots+k+(k+1) = \dfrac{k(k+1)}{2}+(k+1) = \dfrac{k^2+k+2k+2}{2}$\\\vspace{1em}$=\dfrac{(k+1)(k+2)}{2} = \dfrac{(k+1(k+1+1))}{2}$
\end{center}
La \'ultima expresi\'on obtenida es equivalente a $\frac{n(n+1)}{2}$  para un n=k+1, entonces se garantiza que la f\'ormula original satisface para todos los enteros positivos n\textbf{.}




\newpage
\begin{flushleft}
$\displaystyle
b)  \sum_{k=1}^{n}k^{2} = \frac{n(n+1)(2n+1)}{6}$
\end{flushleft}
Reescribimos la expresi\'on:
\begin{center}
$1^2+2^2+\ldots+k^2=\dfrac{n(n+1)(2n+1)}{6}$\\
\end{center}
a) Probamos por inducci\'on con n=1
\begin{center}
$1^2=\dfrac{1(1+1)(2+1)}{6}$\\
\end{center}
Por lo tanto es verdadero ya que 1=1\\\vspace{1em}
\\b) Ahora tomamos un valor arbitrario si n=k,suponiendo que es verdadero entonces:\\ 
\begin{center}
$\displaystyle
\sum_{i=1}^{k}i^{2} = \frac{k(k+1)(2k+1)}{6}$ \\
\end{center}
\begin{flushleft}
c) Ahora con k+1\\
\end{flushleft}
$\displaystyle
\sum_{i=1}^{k+1}i^{2} = \frac{k+1((k+1)+1)(2(k+1)+1)}{6}$ \\
\begin{flushleft}
Ahora aplicamos la hip\'otesis de inducci\'on:
\end{flushleft}
\begin{center}
$\displaystyle
\sum_{i=1}^{k+1}i^{2} = \sum_{i=1}^{k}i^{2} + (k+1)^2$ \\
\end{center}
\begin{flushleft}
Ahora reemplazamos la expresi\'on y desarrollamos:
\end{flushleft}

()


\newpage
\begin{flushleft}
$\displaystyle
c) \sum_{k=1}^{n}k^{3} = \frac{n^2(n+1)^2}{4}$\\
\end{flushleft}
\begin{flushleft}
Reescribimos la expresi\'on:\\
\end{flushleft}
\begin{center}
$1^3+2^3+\ldots+k^3=\dfrac{n^{2}(n+1)^2}{4}$\\
\end{center}
\begin{flushleft}
a) Probamos por inducci\'on con n=1\\
\end{flushleft}
\begin{center}
$1^3=\dfrac{1^{2}(1+1)^2}{4} = \dfrac{1(4)}{4}$\\
\end{center}
\begin{flushleft}
Por lo tanto es verdadero ya que 1=1\\\vspace{1em}
\end{flushleft}
\begin{flushleft}
b) Ahora tomamos un valor arbitrario si n=k,suponiendo que es verdadero entonces:\\
\end{flushleft}
\begin{center}
$\displaystyle
\sum_{i=1}^{k}i^{3} = \frac{k^2(k+1)^2}{4}$\\
\end{center}

c) Ahora con k+1
\begin{center}
$\displaystyle
\sum_{i=1}^{k+1}i^{3} = \frac{(k+1)^2((k+1)+1)^2}{4} = \dfrac{(k+1)^2(k+2)^2}{4}$\\
\end{center}
\begin{flushleft}
Ahora hacemos el supuesto del inciso b)\\
\end{flushleft}

$\displaystyle
\sum_{i=1}^{k+1}i^{3} = \sum_{i=1}^{k}i^{3} + (k+1)^3$\\
\begin{flushleft}
Ahora reemplazamos la expresi\'on y desarrollamos:
\end{flushleft}
$\dfrac{k^2(k+1)^2}{4}+(k+1)^3 = \dfrac{k^2(k+1)^2+4(k+1)^3}{4}
=\dfrac{(k+1)^2(k^2+4k+4)}{4}=\dfrac{(k+1)^2(k+2)^2}{4}$\\\vspace{1em}
\begin{flushleft}
Como llegamos a la misma expresi\'on del inciso b) la propiedad es v\'alida para todo n\'umero natural n\textbf{.}
\end{flushleft}

\newpage
\begin{flushleft}
$\displaystyle\bullet
\sum_{k=1}^{n}k^{4} = \frac{n(n+1)(2n+1)(3n^2+3n-1)}{30}$\\
\end{flushleft}




\newpage
4. Deducir las siguientes igualdades:
\begin{flushleft}
$\displaystyle\bullet
\sum_{k=0}^{n}r^{k} = \dfrac{1-r^{n+1}}{1-r}\vspace{1em}$\\
Podemos reescribir la expresi\'on como:
$\displaystyle
$\begin{center}
$\sum_{k=0}^{n}r^{k} = Sn =r^n+r^{n-1}+r^{n-2}+\ldots+r+1$
\end{center}
Multiplicamos por r\\\vspace{1em}
$rSn = rr^n+rr^{n-1}+rr^{n-2}+\ldots+rr+r$\\
$rSn = r^{n+1}+r^{n}+r^{n-1}+\ldots+r^2+r$\\\vspace{1em}
Ahora restamos Sn-rSn\\\vspace{1em}
$Sn-rSn= r^{n}+r^{n-1}+r^{n-2}+\ldots+r+1-(r^{n+1}+r^n+r^{n-1}+\ldots+r^2+r)$\\
$Sn-rSn=1-r^{n+1}$\\
$Sn(1-r)=1-r^{n+1}$\\\vspace{1em}
$Finalmente:$\\\vspace{1em}
\begin{center}
$Sn = \dfrac{1-r^{n+1}}{1-r}$
\end{center}


%inciso b
\newpage
$\displaystyle\bullet
\sum_{k=0}^{n}kr^{k-1} = \dfrac{1-(n+1)r^{n}+nr^{n+1}}{1-r^{2}}
$
\end{flushleft}



\newpage
5. Calcular las siguientes sumas:\\
\\$\displaystyle a)\dfrac{1}{1*2}+\dfrac{1}{2*3}+\dfrac{1}{3*4}+\ldots+\dfrac{1}{n(n+1)} =\sum_{k=1}^{n}\dfrac{1}{k(k+1)}$\\
\\
\\
$\displaystyle b)\dfrac{1}{1*3}+\dfrac{1}{3+5}+\dfrac{1}{5+7}+\ldots+\dfrac{1}{(2n-1)(2n+1)} =\sum_{k=1}^{n}\dfrac{1}{(2k-1)(2k+1)}$\\
\\






\newpage
\textbf{Secci\'on 1.2}\\
\\1. Demostrar que $a^2+b^2+c^2\geq ab+ac+bc$ para todo a,b,c n\'umeros reales.\\
\\Tenemos lo siguiente\\
\begin{center}
$ (a-b)^2 \geq 0 \rightarrow a^2-2ab+b^2$\\
$ (a-c)^2 \geq 0 \rightarrow a^2-2ac+c^2$\\
$ (b-c)^2 \geq 0 \rightarrow b^2-2bc+c^2$\\
\end{center}
Sumando obtenemos\\
\begin{center}
$2a^2-2ab-2ac+2c^2+2b^2-2bc \geq 0$\\
$2a^2+2b^2+2ab-2ac-2bc \geq 0 $\\
$2a^2+2b^2+2c^2-2(ab+ac+bc) \geq 0 $\\
$2a^2+2b^2+2c^2 \geq 2(ab+ac+bc)$\\
\end{center}
Finalmente\\
\begin{center}
$ a^2+b^2+c^2 \geq ab+ac+bc$.
\end{center}




\newpage
2. Demostrar que si $0 < a < b$ entonces:\\
\begin{center}
$a < \sqrt{ab} < \dfrac{a+b}{2} < b.$
\end{center}
Demostraci\'on.\\
Tenemos que\\
\begin{center}
$(a-b)^2 \geq 0$\\
$(a+b)^2 - 4 ab \geq 0$\\
$(a+b)^2 \geq 4ab$\\
$(a+b)\geq2\sqrt{ab}$\\
$2\sqrt{ab}\leq a+b$\\
\end{center}
Finalmente 
\begin{center}
$\sqrt{ab}\leq \dfrac{a+b}{2}$.
\end{center}



\newpage
3. Demostrar que si $0 < a < b$ entonces:\\
\begin{center}
$b^n-a^n < nb^{n-1}(b-a)\hspace{1em} \forall n \in N. $\\
\end{center}  



\newpage
4. Demostrar lo siguiente:\\ 



\newpage
5. Demostrar lo siguiente:\\ 



\newpage
6. Demostrar lo siguiente:\\ 




\newpage
7. Demostrar que si $a<b$ y $c<d$, entonces $ad+bc<bd+ac$.\\
\\Demostraci\'on\\
\\Tenemos lo siguiente:
\begin{center}
$(b-a)>0 \hspace{1em},\hspace{1em} (d-c)>0$ 
\end{center}
si multiplicamos los correspondientes mayores y los menores , se obtiene:
\begin{center}
$(b-a)(d-c)>0 \Rightarrow bd-ad-bc+ac>0$
\end{center}
ahora sumamos a ambos lados $ad+bc$\\
\begin{center}
$bd+ac>ad+bc$\\
\end{center}
Finalmente
\begin{center}
$ad+bc<bd+ac$.
\end{center}


\newpage
8. Demostrar lo siguiente:\\ 


\newpage
9. Demostrar lo siguiente:\\ 




\newpage
10. Demostrar lo siguiente:\\ 



\newpage
11. Demostrar lo siguiente:\\ 



\newpage
12. Demostrar lo siguiente:\\ 


\newpage
13. Demostrar lo siguiente:\\ 



\newpage
14. Comprobar\\


%inciso a
$\bullet\displaystyle
\sum_{k=1}^{2^n}\dfrac{1}{k} > 1+\dfrac{n}{2}$\\





%inciso b
$\bullet\displaystyle
\sum_{k=1}^{n}\dfrac{1}{\sqrt{k}}> \sqrt{n}$\\
\\
Reescribimos la expresi\'on:
\begin{center}
$\dfrac{1}{\sqrt{1}} + \dfrac{1}{\sqrt{2}}+\dfrac{1}{\sqrt{3}}+\ldots+\dfrac{1}{\sqrt{n}}$
\end{center}
Sea $k\in N$ entonces,\\
\begin{center}
$\sqrt{k} \leq k \Rightarrow k\leq k^2 \Rightarrow 0\leq k^2-k$ \\
$\Rightarrow 0 \leq k(k-1)$
\\$\Rightarrow \dfrac{1}{\sqrt{k}} \geq \dfrac{1}{k}$
\end{center}
y por lo que \\

\begin{center}
$\dfrac{1}{\sqrt{1}} + \dfrac{1}{\sqrt{2}}+\dfrac{1}{\sqrt{3}}+\ldots+\dfrac{1}{\sqrt{n}} \geq \dfrac{1}{\sqrt{n}} + \dfrac{1}{\sqrt{n}}+\dfrac{1}{\sqrt{n}}+\ldots+\dfrac{1}{\sqrt{n}}$
\end{center}

\begin{center}
$\dfrac{1}{\sqrt{1}} + \dfrac{1}{\sqrt{2}}+\dfrac{1}{\sqrt{3}}+\ldots+\dfrac{1}{\sqrt{n}} \geq \dfrac{1}{\sqrt{n}} + \dfrac{1}{\sqrt{n}}+\dfrac{1}{\sqrt{n}}+\ldots+\dfrac{1}{\sqrt{n}}= n(\dfrac{1}{n})=\dfrac{n}{\sqrt{n}}=\sqrt{n}$
\end{center}

\begin{center}
$\displaystyle
\sum_{k=1}^{n}\dfrac{1}{\sqrt{k}} \geq \sqrt{n}$  \\ $\dfrac{1}{\sqrt{k}}>\dfrac{1}{\sqrt{n}}$ con $n>k$.
\end{center}




%Seccion 1.3
\newpage
\textbf{Secci\'on 1.3}\\
\\Teorema:\\
\\a) Si A tiene supremo entonces es \'unico.\\
\\Demostraci\'on\\
Suponemos que $\eta$ y $\theta$ son supremos de A, con	 $\eta  \neq  \theta $.\hspace{1em}Asi que por tricotom\'ia se tiene\\
\begin{center}
$\eta<\theta$ \'o $\eta>\theta$.
\end{center}
Caso $\eta < \theta$,\\entonces $\theta-\eta>0$ y como $\eta$ = SupA, entonces\\
\\i) $\eta \geq$ x para toda x $\in$ A.\\
\\ii) para toda $\epsilon < 0$ existe x $\in$ A,  tal que $\eta-\epsilon<x\leq\eta$. \\
\\Sea $\theta$= SupA, entonces\\
\\
i) $\theta\geq$ x para toda x $\in$ A.\\
ii) para toda $\epsilon > 0$ existe x $\in$ A tal que $\theta-\epsilon<x\leq\theta$\\
\\Sea $\epsilon=\theta-\eta$, entonces \\
\\
$\theta-(\theta-\eta)<x$\\
$\eta<x$ "contradicci\'on"\\
por lo tanto $\theta=\eta $ es \'unico.

\newpage
Teorema. El conjunto de los n\'umeros naturales no est\'a acotado superiormente.\\
\\Demostraci\'on\\
\\Sea N conjunto de los n\'umeros naturales y suponiendo que est\'a acotado superiormente, siendo N no vac\'io ya que 1 $\in N$, entonces por la propiedad del supremo.\\
\\Como $\zeta = SupN$ se tiene:\\
\\i) $\zeta\geq n$ para todo n $\in N$ ya que $\zeta$ es cota superior;\\
\\ii) para todo $\epsilon>0$ existe m $\in N$ tal que 
\begin{center}
$ \zeta-\epsilon<m$
\end{center}
Si $\epsilon=1>0$ existe k $\in N$ tal que\\
\begin{center}
$\zeta-1<k$\\
$\zeta<k+1 \in N$
\end{center}
lo que contradice el hecho i).

\newpage
Teorema (Propiedad Arquimediana). Para x,y n\'umeros reales con $x>0$, existe \textit{n} entero positivo tal que
\begin{center}
$nx>y$
\end{center}
Demostraci\'on\\
Por tricotom\'ia para $x,y \in R$ se tiene
\begin{center}
 $ x < y $ , $x = y $ \'o $ x > y $
\end{center}
Caso $x>y$,\hspace{1em}sea $n\geq 1$, $nx > y$.\\
Caso $x=y$,\hspace{1em}sea $n\geq 2$ y se tiene $nx>y$.\\
Caso $x<y$,\hspace{1em}$0<x<y$.\\
\\Sea A=\{$nx \lvert$ $n\in N\}$. A $\neq$ $\emptyset$
ya que $1x\in A$ y suponer que para toda $n\in N$ $nx\leq y$, con lo que se tiene que A est\'a acotado superiormente. Por la propiedad del supremo A tiene supremo, sea $\zeta=SupA.$\\
\\Siendo $\zeta=SupA$\\
\\
i) $\zeta\geq nx$ para todo $ n \in N$.\\
ii) para todo $\epsilon>0$ existe $mx \in A$ tal que $\zeta-\epsilon<mx$.\\
\\Sea $\epsilon=x>0$ entonces existe $kx \in A$ tal que
\begin{center}
$\zeta-x<kx$\\
$\zeta<kx+x$
$\zeta<(k+1)x \in A$\\
\end{center}
lo que contradice i).









\end{document}
 
