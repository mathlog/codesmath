\documentclass[]{article}
\usepackage[spanish]{babel}
    \usepackage{tikz}
%opening
\title{Tarea 1 Probabilidad\\}
\author{Adri\'an Mendoza Zamora
	\\
	Prof. Adri\'an Alc\'antar Torres}

\begin{document}

\maketitle
\begin{center}
\textbf{	Ley aditiva de la probabilidad}
\end{center}

Sean A y B dos eventos cualesquiera de un espacio muestral $\Omega$ :\\

\begin{center}
	\pagestyle{empty}

\def\firstcircle{(0,0) circle (1.5cm)}
\def\thirdcircle{(0:2cm) circle (1.5cm)}

\begin{tikzpicture}[line width=0.25pt]

\begin{scope}

\fill[white] \firstcircle;
\fill[white] \thirdcircle;
\end{scope}

\begin{scope}
\clip \firstcircle;
\fill [white] \thirdcircle;

\end{scope}


\draw \firstcircle node[text=black, below] {$A$};
\draw \thirdcircle node [text=black,below] {$B$};

\end{tikzpicture}

\end{center}

Entonces tenemos lo siguiente:\\

\begin{center}
$A=(A-B) \cup (A \cap B)\hspace{1cm} \Rightarrow\hspace{1cm}P(A)=P(A-B)+P(A \cap B) $\\
$B=(B-A) \cup (B \cap A) \hspace{1cm} \Rightarrow\hspace{1cm} P(B)=P(B-A)+P(B \cap A)$\\
\end{center}
\begin{center}
	$P(A-B)=P(A)-P(A \cap B)$\\
$P(B-A)=P(B)-P(B \cap A)$\\
\end{center}


\begin{flushleft}
	$P(A \cup B)=P(A-B)+ P(A \cap B) + P(B-A)$,  as\'i
\end{flushleft}
$P(A \cup B)=P((A)-P(A \cap B)) + P(A \cap B) + (P(B)-P(A \cap B))$, por lo tanto \\\\
$P(A \cup B)=P(A)+P(B)-P(A \cap B). $



\end{document}
